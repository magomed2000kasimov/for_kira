\CWHeader{Лабораторная работа \textnumero 9}


\CWProblem{
Разработать жадный алгоритм решения задачи, определяемой своим вариантом. Оценить сложность по времени и объём затрачиваемой оперативной памяти.

\textbf{Вариант задачи} \\
Задан неориентированный двудольный граф, состоящий из $n$ вершин и $m$ ребер. Вершины пронумерованы целыми числами от 1 до $n$. Необходимо найти максимальное паросочетание в графе алгоритмом Куна. Для обеспечения однозначности ответа списки смежности графа следует предварительно отсортировать. Граф не содержит петель и кратных ребер.


\textbf{Формат входных данных}\\
В первой строке заданы $1 \leq n \leq 110000$ и $1 \leq m \leq 40000$. В следующих $m$ строках записаны ребра. Каждая строка содержит пару чисел – номера вершин, соединенных ребром.

\textbf{Формат результата} \\
В первой строке следует вывести число ребер в найденном паросочетании. В следующих строках нужно вывести сами ребра, по одному в строке. Каждое ребро представляется парой чисел – номерами соответствующих вершин. Строки должны быть отсортированы по минимальному номеру вершины на ребре. Пары чисел в одной строке также должны быть отсортированы.
}

\pagebreak
